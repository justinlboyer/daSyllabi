\documentclass[12pt]{article}

\usepackage{hyperref}


\textwidth=7in
\textheight=9.5in
\topmargin=-1in
\headheight=0in
\headsep=.5in
\hoffset  -.85in

\pagestyle{empty}

\renewcommand{\thefootnote}{\fnsymbol{footnote}}
\begin{document}

\begin{center}
{\bf MATH 3 Honors\ Sec. 1  \ \ MWF 8:00 - 9:00 AM,  Room:    
}
\end{center}

\setlength{\unitlength}{1in}

\begin{picture}(6,.1) 
\put(0,0) {\line(1,0){6.25}}         
\end{picture}

 

\renewcommand{\arraystretch}{2}

\vskip.25in
\noindent\textbf{Instructor:} Justin Boyer, email: TBD, Phone: 555-5555, website: \url{https://sites.google.com/site/justinlboyer/}
\vskip.25in
\noindent\textbf{Office Hours:} 8:30-9:20 AM, 2:00-3:00 PM TT
and by  appointment.

\vskip.25in
\noindent\textbf{Textbook:}  Put your textbook information here.

\vskip.25in
\noindent\textbf{Prerequisites:}%\footnotemark
A list of prerequisites.


%\footnotetext{Footnote text goes here.}

\vspace*{.15in}

\noindent \textbf{Course Outline:} 

\begin{center} \begin{minipage}{5in}
\begin{flushleft}
Chapter 1 \dotfill approx 3 days\\
Chapter 2 \dotfill approx 3 days\\
Chapter 3 \dotfill approx 2 1/2 days\\
First Exam  \dotfill 1 day\\
Chapter 4 \dotfill approx 2 days\\
Chapter 8 \dotfill approx 1 1/2 days\\
Chapter 9 \dotfill approx 1 1/2 days\\
Chapter 5 \dotfill approx 2 days\\
Second Exam \dotfill 1 day\\
Chapter 6 \dotfill approx 2 days \\
Chapter 7 \dotfill approx 1 days\\
Chapter 11 \dotfill approx 1 1/2 days\\
Chapter 12 \dotfill approx 2 days\\
Third Exam \dotfill 1 day\\
\end{flushleft}
\end{minipage}
\end{center}

\vspace*{.15in}
\noindent\textbf{Grade Policy:} I grade based on mastery of standards.  This means that you (the student) are graded based on your level of proficiency of standards taken from the core curriculum as well as the standards I develop.  If you do not show me you have learned the standards you will not pass this class.   Each standard is graded on a scale $0-9$, each number represents a students level of proficiency:
\begin{itemize}
\item A 9 is mastery.
\item An 8 is mastery, however the student made a small mistake, likely with arithmetic, for example: $2+1=4$.
\item A 7 is a conceptual misunderstanding.
\item A 6 is a fundamental misunderstanding, the student likely needs to be retaught the material.
\item 5 and below represents various levels of attempts but all neglect any understanding of the material involved.
\end{itemize}
If you have mastered the material and you continue to exhibit proficiency of the material you will earn extra credit in the class, a 10 for that specific standard.\\

If $ \geq 80\%$ of the class is at an 8 or higher I will not reteach the material.  If $\geq 20\%$ of the class is at a 7 or below I will (at a minimum) do a quick review of the standard in question.\\

I reserve the right to include any "standard'' I deem necessary to facilitate your development as mathematicians.  Including but not limited to the following standards:
\begin{itemize}
\item The student follows directions the first time.
\item The student completes all homework.
\item The student writes in a neat, organized manner.
\item The student shows sufficient work.
\item The student immediately begins the warm up upon entering the classroom.
\end{itemize}


\vskip.25in
\noindent \textbf{Course Objectives}:  First and foremost the objective of this course is to bring out the mathematician in you.  It is my goal for you to leave this course identifying yourself as a mathematician.  This will benefit you throughout your life, by helping you destroy the barriers you create for yourself.\\

The academic goal for this course is to

\vskip.25in
\noindent\textbf{Grading}: Your grade will fluctuate during this class.  It is possible that at some point you will have a grade that is not to your liking.  The grades are very fluid and are a snapshot of your progress on any of the given standards at that moment.  Therefore it is important that you frequently check your grades to see what standards you should be focusing on.\\
\textbf{Homework}  You will have daily homework assignments, these will be graded for completeness not for correctness.  It is your responsibility to ask questions regarding the homework so that you completely understand each problem.\\
\textbf{Project}  Each student will participate in a project which will span the entire trimester.  The project will culminate in a research paper and a presentation.  The grade earned will be based on a rubric, which will be handed out along with possible project topic list.\\
\textbf{Quizzes} We will have at least one cumulative quiz a week.  During which I will asses your level on each of the standards covered by the quiz. \\
\textbf{Exams} There will be two exams each trimester.  The exams will be comprehensive.  Each problem will align with one or more standard, your updated grade for each standard will be based on the median score you receive on each standard.

%\vskip.25in
%\noindent\textbf{Homework}:  A Description of the homework policy.

\vskip.25in
\noindent\textbf{Academic Honesty}:  As a Brighton High School student, you have agreed to abide by the
School�s academic honesty policy, and the Student Honor Code.
Lack of knowledge of the academic honesty policy is not a
reasonable explanation for a violation. Questions related to course assignments and the
academic honesty policy should be directed to the instructor.

\vskip.25in
\noindent\textbf{Extra Help}:  Dot not hesitate to come to class during office hours or by appointment
to discuss a homework problem or any aspect of the course. You also may want to consider using one of the great online sources such as \url{https://www.ixl.com/math/} or \url{https://www.khanacademy.org/} 

\vskip.25in
\noindent\textbf{Brighton High School Attendance Policy}:  \url{http://brightonhigh.canyonsdistrict.org/docs/2015-16_BHS_Attendace_Policy.docx}
Students are expected to attend classes regularly. A student who incurs an excessive
number of absences will be required to attend a conference including the students guardians and possibly administration.

\vskip.25in
\noindent\textbf{Important Dates}:
\begin{center} \begin{minipage}{5in}
\begin{flushleft}
%Drop Deadline \dotfill Month Day\\
%Add Deadline \dotfill Month Day\\
Midterm \dotfill Month Day\\
%Second Test  \dotfill Month Day\\
%Third Test \dotfill Month Day\\
Project Deadline/Presentations \dotfill Month Day\\
Course Final \dotfill Month Day\\
\end{flushleft}
\end{minipage}
\end{center}








\end{document}